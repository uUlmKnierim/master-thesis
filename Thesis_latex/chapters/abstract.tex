% !TEX root = ../arbeit.tex
\chapter*{Abstract}
Providing instant access to information when and wherever it is desired is one of the aims of the ubiquitous computing research area. To accomplish omnipresent visualisation of information, display technology needs to be present everywhere. As modern technology is not yet mature enough for truly ubiquitous visualisation, researchers instead emulate the rendering of information on any surface by use of a \ac{PROCAMS}. These devices consist of a projector to render content onto any surface, and a camera to obtain user interaction.

This thesis proposes a novel PROCAMS for effortless domestic deployment. Its key features are a user-friendly interface, automatic calibration and the stand-alone character of the unit, as well as the ability to rotate along two axes, providing two degrees of freedom. As the proposed system is ceiling-mounted, it is able to transform every surface visible from this spot into a projected touch screen. Content is encapsulated into widgets which are pre-warped for a rectified presentation to the user. Rapid development of widgets is accomplished by decoupling the complexity of spatially and geometrically aware projection from the widget.
Hence touch detection is also dissociated, interaction cues are injected into widgets.

The objective of this thesis is to integrate results of current research with several new techniques and small inexpensive hardware to create a small, rotatable, calibration-free projector camera system. This yields an effortlessly deployable prototype which for the first time allows the evaluation of new interaction concepts and everywhere information projection directly in users' well-known environments; their homes.