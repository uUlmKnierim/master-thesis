% !TEX root = ../arbeit.tex
\chapter{Conclusion}\label{chapter:conclusion}
In the course of this thesis, a \acl{PROCAMS} was designed, built and evaluated. The built PROCAMS can be deployed in domestic environments to make research beyond the laboratory possible. Previously published PROCAMS were heavy, bulky and required manual calibration before touch interaction and rectified projection of content was possible. In contrast, in this thesis a light, stand-alone and rotatable prototype was developed which does not require any calibration to be executed by the user.

The proposed PROCAMS consists of a PrimeSense Carmine 1.09 depth sensing camera for touch input detection, a small light-intensive projector as well as a powerful \acl{SBC}, the ODROID-XU. The volume of composed parts is less than \SI{1}{\dm\cubed} and the whole unit weighs less than \SI{1}{\kg}. All parts are attached to a small inexpensive self-built pan-tilt unit. The unit is ceiling-mounted and enables the PROCAMS to project information in any direction. As an input method, a light-weight touch detection algorithm was implemented, using images of the depth sensing camera combined with a tracking algorithm.

As part of the designing process semi-structured interviews were conducted. Eighteen potential end users were visited at their homes and realistic setups were built with a PROCAMS mock-up, capable of projecting static user interfaces. Participants used the mock-up in particular to replace already existing technical things or displays such as wall paintings, televisions, clocks, printed bus timetables or a set of remote controls. They strongly preferred to project onto flat surfaces even after clarifying that it would be possible to project without distortion to irregular surfaces.

A generic software and hardware architecture design was elaborated. Creating a configuration free, movable PROCAMS offering touch input on arbitrary surfaces was the main focus while developing a small and light system from scratch. For displaying graphical user interface elements, the Qt library was used as it supports rapid prototyping. Several widgets were implemented for different scenarios such as an image viewer, bus timetable or news browser. Due to the software architecture, already existing Qt widgets are effortlessly  integrated into the system. 
Implementing a pre-warping technique allows projection onto surfaces from any projector alignment so that it appears correct to the observer. Therefore, a spatial model based on the depth sensing camera is calculated on demand. The movement of the PROCAMS is controlled via a remote. An Android App containing a virtual joystick enables the user to control the pan-tilt unit. Furthermore, the App allows the user to store and reload previous configurations.

Finally, a technical evaluation was conducted. The accuracy of touch input as well as the pace and fidelity of the built prototype were evaluated. The built PROCAMS provides a very reasonable touch accuracy. Considering its limited processing power, the prototype has a good overall performance and users were able use it in an enjoyable way.

\section{Challenges}
Using inexpensive hardware makes an extensive deployment possible. Unfortunately, the quality and performance of the used hardware are not particular valuable. Especially the used depth sensing camera provides only very noisy images. One big challenge was to filter the noise to enhance the image quality, as this directly correlates with improved touch accuracy and reduced false touch events. A huge hurdle was to accomplish this filtering as well as the touch detection, while keeping the time between the actual touch and reaction by the system small. The limited processing resources make this task even more challenging.

Another challenge was the development of the hardware construction due to a lack of previous experience in designing and printing parts as well as the control and proper use of actuators. It took several attempts until the design was robust and well balanced enough that the unit operates in a decent manner.
Furthermore, there were many other minor and major challenges such as camera calibration or transformation of touch events, until the system finally manifested in its current accurate and usable form. 

\section{Future Work}

A basic platform for everywhere information rendering and touch interaction was created. However, there are several areas which remain open to further development. On the hardware side, some components could be replaced to gain a higher performance and less noise. The servo responsible for projector focus should be replaced, as it is very loud. Furthermore, replacing the servos of the tilt-pan platform by high precision servos would accomplish a higher accuracy when approaching a commanded alignment. In a final step, the fan of the \ac{SBC} could be replaced to obtain an almost silent operation of the PROCAMS. 

On the software side, one should consider which input methods should be supported in planned long-term studies. Only four simple touch interactions were implemented, but the used hardware is capable of providing much more interaction possibilities. The implemented framework allows for an easy integration of new interaction or input methods. Several more touch interactions like multi-touch or swiping gestures are conceivable. Colour, size or quantity of touches as well as the expansion from surface interaction into free space interaction could be used for interesting new interaction concepts. As mentioned in the interviews, spoken commands could easily be integrated. A stereo microphone is already built-in which allows for a coarse localisation of the audio source. Another idea is to link the PROCAMS with the smart phone of the user to allow fast remote text input when speech recognition is not feasible.

The proposed PROCAMS finally allows the conduction of large long-term user studies in users' homes. Due to the inexpensive cost of production of the proposed PROCAMS, the accommodations could be equipped with several units. For the first time, it could be examined how everywhere displays are used in a domestic environment. Primarily the use of widgets and the preferred location for information placement, as well as interaction methods, could be evaluated. Another crucial aspect which could be reappraised is the social effect of everywhere projection, including its influence on privacy concerns. Furthermore, it would allow an examination of  the most appropriate input method, how the movability of the system is used, and how users would interact if real world objects are augmented with information. 

It is my hope that the proposed hardware and software framework can support the discovery of answers to these questions by enabling an easy deployment of the PROCAMS as well as the rapid application development for everywhere projections. 