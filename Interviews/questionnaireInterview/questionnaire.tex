\documentclass[a4paper,10pt,BCOR10mm,oneside,headsepline]{scrartcl}
\usepackage[ngerman]{babel}
\usepackage[utf8]{inputenc}
\usepackage{wasysym}% provides \ocircle and \Box
\usepackage{enumitem}% easy control of topsep and leftmargin for lists
\usepackage{color}% used for background color
\usepackage{forloop}% used for \Qrating and \Qlines
\usepackage{ifthen}% used for \Qitem and \QItem
\usepackage{typearea}
\areaset{17cm}{26cm}
\setlength{\topmargin}{-1cm}
\usepackage{scrpage2}
\pagestyle{scrheadings}
\ihead{Interaktives Projektor-Kamera-System}
\ohead{\pagemark}
\chead{}
\cfoot{}



%% Personal changes 
\usepackage{setspace}


% Raised Rule Command:
%  Arg 1 (Optional) - How high to raise the rule
%  Arg 2            - Thickness of the rule
\newcommand{\Qeline}{\leaders\hbox{\rule[-1pt]{1pt}{0.6pt}}\hfill}

%% end personal changes 


%%%%%%%%%%%%%%%%%%%%%%%%%%%%%%%%%%%%%%%%%%%%%%%%%%%%%%%%%%%%
%% Beginning of questionnaire command definitions         %%
%%%%%%%%%%%%%%%%%%%%%%%%%%%%%%%%%%%%%%%%%%%%%%%%%%%%%%%%%%%%
%%
%% 2010, 2012 by Sven Hartenstein
%% mail@svenhartenstein.de
%% http://www.svenhartenstein.de
%%
%% Please be warned that this is NOT a full-featured framework for
%% creating (all sorts of) questionnaires. Rather, it is a small
%% collection of LaTeX commands that I found useful when creating a
%% questionnaire. Feel free to copy and adjust any parts you like.
%% Most probably, you will want to change the commands, so that they
%% fit your taste.
%%
%% Also note that I am not a LaTeX expert! Things can very likely be
%% done much more elegant than I was able to. If you have suggestions
%% about what can be improved please send me an email. I intend to
%% add good tipps to my website and to name contributers of course.
%%
%% 10/2012: Thanks to karathan for the suggestion to put \noindent
%% before \rule!

%% \Qq = Questionaire question. Oh, this is just too simple. It helps
%% making it easy to globally change the appearance of questions.
\newcommand{\Qq}[1]{\textbf{#1}}

%% \QO = Circle or box to be ticked. Used both by direct call and by
%% \Qrating and \Qlist.
\newcommand{\QO}{$\Box$}% or: $\ocircle$

%% \Qrating = Automatically create a rating scale with NUM steps, like
%% this: 0--0--0--0--0.
\newcounter{qr}
\newcommand{\Qrating}[1]{\QO\forloop{qr}{1}{\value{qr} < #1}{---\QO}}

%% \Qline = Again, this is very simple. It helps setting the line
%% thickness globally. Used both by direct call and by \Qlines.
\newcommand{\Qline}[1]{\noindent\rule[-2pt]{#1}{0.6pt}}

%% \Qlines = Insert NUM lines with width=\linewith. You can change the
%% \vskip value to adjust the spacing.
\newcounter{ql}
\newcommand{\Qlines}[1]{\forloop{ql}{0}{\value{ql}<#1}{\vskip0.5em\Qline{\linewidth}}}

%% \Qlist = This is an environment very similar to itemize but with
%% \QO in front of each list item. Useful for classical multiple
%% choice. Change leftmargin and topsep accourding to your taste.
\newenvironment{Qlist}{%
  \renewcommand{\labelitemi}{\QO}
  \begin{itemize}[leftmargin=1.5em,topsep=-.5em]
}{%
  \end{itemize}
}

%% \Qtab = A "tabulator simulation". The first argument is the
%% distance from the left margin. The second argument is content which
%% is indented within the current row.
\newlength{\qt}
\newcommand{\Qtab}[2]{
  \setlength{\qt}{\linewidth}
  \addtolength{\qt}{-#1}
  \hfill\parbox[t]{\qt}{\raggedright #2}
}

%% \Qitem = Item with automatic numbering. The first optional argument
%% can be used to create sub-items like 2a, 2b, 2c, ... The item
%% number is increased if the first argument is omitted or equals 'a'.
%% You will have to adjust this if you prefer a different numbering
%% scheme. Adjust topsep and leftmargin as needed.
\newcounter{itemnummer}
\newcommand{\Qitem}[2][]{% #1 optional, #2 notwendig
  \ifthenelse{\equal{#1}{}}{\stepcounter{itemnummer}}{}
  \ifthenelse{\equal{#1}{a}}{\stepcounter{itemnummer}}{}
  \begin{enumerate}[topsep=2pt,leftmargin=2.8em]
  \item[\textbf{\arabic{itemnummer}#1.}] #2
  \end{enumerate}
}

%% \QItem = Like \Qitem but with alternating background color. This
%% might be error prone as I hard-coded some lengths (-5.25pt and
%% -3pt)! I do not yet understand why I need them.
\definecolor{bgodd}{rgb}{0.8,0.8,0.8}
\definecolor{bgeven}{rgb}{0.9,0.9,0.9}
\newcounter{itemoddeven}
\newlength{\gb}
\newcommand{\QItem}[2][]{% #1 optional, #2 notwendig
  \setlength{\gb}{\linewidth}
  \addtolength{\gb}{-5.25pt}
  \ifthenelse{\equal{\value{itemoddeven}}{0}}{%
    \noindent\colorbox{bgeven}{\hskip-3pt\begin{minipage}{\gb}\Qitem[#1]{#2}\end{minipage}}%
    \stepcounter{itemoddeven}%
  }{%
    \noindent\colorbox{bgodd}{\hskip-3pt\begin{minipage}{\gb}\Qitem[#1]{#2}\end{minipage}}%
    \setcounter{itemoddeven}{0}%
  }
}

%%%%%%%%%%%%%%%%%%%%%%%%%%%%%%%%%%%%%%%%%%%%%%%%%%%%%%%%%%%%
%% End of questionnaire command definitions               %%
%%%%%%%%%%%%%%%%%%%%%%%%%%%%%%%%%%%%%%%%%%%%%%%%%%%%%%%%%%%%

\begin{document}

\begin{center}
  \textbf{\huge Interatives Projektor-Kamerea-System \vskip0.2em Beobachtung}
\end{center}

\vspace{-1em}


%\noindent Welcome to this very important survey with which we
%researchers want to look deep inside of you (be afraid!). Anyway,
%thank you for filling it all out.
%
%\noindent \textit{Please note that no tabular environment is used in
%  this example questionnaire. Of course, you could use tabular to
%  create more complex layout.}

\onehalfspacing

\section*{Raum: Küche}
\vspace{-1em}
\Qitem{ \Qq{Art der Platzierung:}\Qtab{7cm}{ \Qline{9cm}}}
\Qitem{ \Qq{Höhe der Projektor-Kamera Einheit:}\Qtab{7cm}{ \Qline{9cm}}}
\Qitem{ \Qq{Orientierung der Projektor-Kamera Einheit:} \Qtab{8cm}{ \Qline{8cm}}}
\Qitem{ \Qq{Abstand Projector - Surface:} \Qtab{7cm}{ \Qline{9cm}}}
\Qitem{ \Qq{Kombination mehrere Einheiten:} \Qtab{7cm}{ \Qline{9cm}}}
\Qitem{ \Qq{Welche Informationen würden angezeigt werden:} \Qtab{9cm}{ \Qline{7cm}}}
\Qitem{ \Qq{Gewünschte input Methode: \Qtab{7cm}{ \Qline{9cm}}}}
\Qitem{ \Qq{Erzeugen von UIs: \Qtab{7cm}{ \Qline{9cm}}}}


\Qitem{ \Qq{Beobachtungen:}\Qlines{6}}


\section*{Raum: Wohnzimmer}
\vspace{-1em}
\Qitem{ \Qq{Art der Platzierung:}\Qtab{7cm}{ \Qline{9cm}}}
\Qitem{ \Qq{Höhe der Projektor-Kamera Einheit:}\Qtab{7cm}{ \Qline{9cm}}}
\Qitem{ \Qq{Orientierung der Projektor-Kamera Einheit:} \Qtab{8cm}{ \Qline{8cm}}}
\Qitem{ \Qq{Abstand Projector - Surface:} \Qtab{7cm}{ \Qline{9cm}}}
\Qitem{ \Qq{Kombination mehrere Einheiten:} \Qtab{7cm}{ \Qline{9cm}}}
\Qitem{ \Qq{Welche Informationen würden angezeigt werden:} \Qtab{9cm}{ \Qline{7cm}}}
\Qitem{ \Qq{Gewünschte input Methode:} \Qtab{7cm}{ \Qline{9cm}}}
\Qitem{ \Qq{Erzeugen von UIs: \Qtab{7cm}{ \Qline{9cm}}}}



\Qitem{ \Qq{Beobachtungen:}\Qlines{6}}


\section*{Raum: Bad}
\vspace{-1em}

\Qitem{ \Qq{Art der Platzierung:}\Qtab{7cm}{ \Qline{9cm}}}
\Qitem{ \Qq{Höhe der Projektor-Kamera Einheit:}\Qtab{7cm}{ \Qline{9cm}}}
\Qitem{ \Qq{Orientierung der Projektor-Kamera Einheit:} \Qtab{8cm}{ \Qline{8cm}}}
\Qitem{ \Qq{Abstand Projector - Surface:} \Qtab{7cm}{ \Qline{9cm}}}
\Qitem{ \Qq{Kombination mehrere Einheiten:} \Qtab{7cm}{ \Qline{9cm}}}
\Qitem{ \Qq{Welche Informationen würden angezeigt werden:} \Qtab{9cm}{ \Qline{7cm}}}
\Qitem{ \Qq{Gewünschte input Methode:} \Qtab{7cm}{ \Qline{9cm}}}
\Qitem{ \Qq{Erzeugen von UIs: \Qtab{7cm}{ \Qline{9cm}}}}



\Qitem{ \Qq{Beobachtungen:}\Qlines{7}}


\section*{Raum: \Qline{8cm}}
\vspace{-1em}

\Qitem{ \Qq{Art der Platzierung:}\Qtab{7cm}{ \Qline{9cm}}}
\Qitem{ \Qq{Höhe der Projektor-Kamera Einheit:}\Qtab{7cm}{ \Qline{9cm}}}
\Qitem{ \Qq{Orientierung der Projektor-Kamera Einheit:} \Qtab{8cm}{ \Qline{8cm}}}
\Qitem{ \Qq{Abstand Projector - Surface:} \Qtab{7cm}{ \Qline{9cm}}}
\Qitem{ \Qq{Kombination mehrere Einheiten:} \Qtab{7cm}{ \Qline{9cm}}}
\Qitem{ \Qq{Welche Informationen würden angezeigt werden:} \Qtab{9cm}{ \Qline{7cm}}}
\Qitem{ \Qq{Gewünschte input Methode:} \Qtab{7cm}{ \Qline{9cm}}}
\Qitem{ \Qq{Erzeugen von UIs: \Qtab{7cm}{ \Qline{9cm}}}}



\Qitem{ \Qq{Beobachtungen:}\Qlines{7}}


\end{document}